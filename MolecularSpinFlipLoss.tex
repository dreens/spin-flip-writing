	% ****** Start of file MolecularSpinFlipLoss.tex ******
%
%
%

\documentclass[%
 reprint,
%superscriptaddress,
groupedaddress,
%unsortedaddress,
%runinaddress,
%frontmatterverbose, 
%preprint,
%showpacs,preprintnumbers,
%nofootinbib,
%nobibnotes,
%bibnotes,
 amsmath,amssymb,
 aps,
prl,
%pra,
%prb,
%rmp,
%prstab,
%prstper,
%floatfix,
]{revtex4-1}

\usepackage{graphicx}% Include figure files
\usepackage{dcolumn}% Align table columns on decimal point
\usepackage{bm}% bold math
\usepackage[hidelinks]{hyperref}% add hypertext capabilities
%\usepackage[mathlines]{lineno}% Enable numbering of text and display math
%\linenumbers\relax % Commence numbering lines

%\usepackage[showframe,%Uncomment any one of the following lines to test 
%%scale=0.7, marginratio={1:1, 2:3}, ignoreall,% default settings
%%text={7in,10in},centering,
%%margin=1.5in,
%%total={6.5in,8.75in}, top=1.2in, left=0.9in, includefoot,
%%height=10in,a5paper,hmargin={3cm,0.8in},
%]{geometry}


\newcommand{\epb}{{$E\!\perp\!B$}}
\newcommand{\epbm}{{E\!\perp\!B}}

\begin{document}

%\preprint{APS/123-QED}

\title{Molecular Spin-Flip Loss and a Dual Quadrupole Trap}%

\author{David Reens}
\altaffiliation{dave.reens@colorado.edu}%
\author{Hao Wu}
\author{Tim Langen}%
\altaffiliation{Present Address: 5. Physikalisches Institut and Center for Integrated Quantum Science and Technology (IQST), Universit\"at Stuttgart, Pfaffenwaldring 57, 70569 Stuttgart, Germany}
\author{Jun Ye}
\affiliation{%
JILA, National Institute of Standards and Technology
and the University of Colorado \\ Department of Physics, University
of Colorado, Boulder, Colorado 80309-0440, USA
}%

\date{\today}% It is always \today, today,


%%%%%%%%%%%%%%%%%%%%%
%ABSTRACT
%%%%%%%%%%%%%%%%%%%%%
\begin{abstract}
Doubly dipolar molecules exhibit complex internal spin-dynamics when electric and magnetic fields are both applied. Near magnetic trap minima, these spin-dynamics lead to enhancements in Majorana spin-flip transitions by many orders of magnitude relative to atoms, and are thus an important obstacle for progress in molecule trapping and cooling. The effect is strongest for Hund's case (a) states but is quite significant for Hund's case (b) as well. We study these internal spin-dynamics with OH molecules and devise a trap geometry where spin-flip loss can be tuned from over $200 \text{ s}^{-1} $ to complete removal with only a weak external bias coil and with no sacrifice of trap strength.
%A new electromagnetic trap geometry allows full tuning of complex molecular spin-dynamics in crossed electric and magnetic fields. If not tuned properly, these dynamics lead to spin-flip loss that afflicts a wide set of candidate molecules. The spin-flip loss can be significant even above $100\text{ mK}$ and varies inversely with temperature, so its removal represents a critical step toward ultracold molecules. The trapping geometry features a $0.5 \text{ K}$ trap depth and $5 \text{ T/cm}$ trap strength, and allows spin-dynamics to be tuned with a weak external bias coil. Spin-flip loss is tuned in a $170 \text{ mK}$ sample of OH molecules from over $200 \text{ s}^{-1} $ to below the vacuum limited lifetime of $2 \text{ s}^{-1}$.
\end{abstract}


\maketitle


%%%%%%%%%%%%%%%%%%%%%%%%%%%%%%%%%
%
%     III   NNN   TTT   RRR   OOO   DDD   UUU   CCC   TTT   III   OOO   NNN
%     III   NNN   TTT   RRR   OOO   DDD   UUU   CCC   TTT   III   OOO   NNN
%     III   NNN   TTT   RRR   OOO   DDD   UUU   CCC   TTT   III   OOO   NNN
%
%%%%%%%%%%%%%%%%%%%%%%%%%%%%%%%%%
%\section{Introduction}
The ultracold regime extends toward molecules on many fronts~\cite{Carr2009}. KRb molecules have reached lattice quantum degeneracy~\cite{Moses2015} and other bialkalis continue to progress~\cite{Takekoshi2014, Park2015}. Creative and carefully engineered laser cooling strategies are tackling certain nearly vibrationally diagonal molecules~\cite{Hummon2013, Barry2014, Zhelyazkova2014, Steinecker2016, Hemmerling2016}. A diverse array of alternative strategies have succeeded to greater or lesser extents on other molecules~\cite{Doyle1998, Bethlem1999, Bochinski2003, Narevicius2008, Wiederkehr2012, Prehn2016}. All of these molecules will require secondary strategies like evaporation or sympathetic cooling to make further gains in phase space density~\cite{Parazzoli2011, Stuhl2012evap, Quemener2016}. They also may face a familiar challenge: spin flip loss near the zero of a magnetic trap, but dramatically enhanced for many doubly dipolar molecules due to their internal spin dynamics in mixed electric and magnetic fields. 

The knowledge of spin flips or Majorana hops as an eventual trap lifetime limit predates the very first magnetic trapping of neutrals~\cite{Migdall1985}. Spin flips were directly observed near $50\,\mu\text{K}$ and overcome with a time-orbiting potential trap~\cite{Petrich1995} and a plugged dipole trap~\cite{Davis1995}, famously enabling the first production of Bose-Einstein condensates. In our earlier investigations, we observed loss of magnetically trapped hydroxyl radicals (OH) with applied electric field \cite{Stuhl2012uwave}. This trap loss occurred for sub-states of OH's $\mathrm{X}^2\Pi_{J=3/2}$ ground state manifold other than the doubly weak-field seeking one of positive parity and full spin polarization, the $|f,m_J]3/2\rangle$ state, colored blue in Fig.~\ref{fig:blocking}. These other states eventually intersect levels of opposite parity at non-zero magnetic fields, where electric field can then open avoided crossings leading to trap loss. We now identify internal spin-dynamics leading to trap loss near zero magnetic field for the $|f,3/2\rangle$ state that operates even in the $50\text{ mK}$ range and below. 

%We first suspected this during experiments in our previous trap geometry, a 3D permanent magnet quadrupole trap with homogeneous electric field \cite{Stuhl2013}, but it was difficult to disentangle it from competing effects.  Our new trap geometry addresses this, but we use our previous geometry as our starting point to explain the internal spin-dynamics that lead to the enhanced loss.

\begin{figure}[tb]
\includegraphics[width=\linewidth]{Blocking/blocking_out.png}%
\caption{
(a) Four Zeeman split lines in OH's $J=3/2$ ground manifold, with the well-trapped state in blue and its spin-flip partner in red. (b) Again with $E=150\text{V/cm}$, and $E$\raisebox{0.5px}{$\parallel$}$B$. (c) Again with \epb. Note vastly reduced red-blue splitting. Other angles in appendix of~\cite{Stuhl2013}. (d) The opposite parity manifold of electrically strong field seeking substates sits $80\text{ mK}$ below, this is the lambda doublet splitting $\Delta$. (e) Energy splitting contours every 2 mK near the zero of a magnetic quadrupole trap with $2\text{ T/cm}$ gradient~\cite{Stuhl2012uwave}. B-field vectors in blue. (f) Again with $E=150\text{ V/cm}$. Note drastic widening of lowest contour (red). Vector direction gives the effective quantization axis of the trapped state, $\mu_BB\pm d_EE$ above (below) the horizontal centerline. Vector magnitude gives potential energy relative to trap center.
\label{fig:blocking}}
\end{figure}

These internal spin-dynamics are subtle, having eluded three previous investigations of note: In~\cite{Lara2008} the analogues of atomic spin-flip loss for molecules in mixed fields were modeled, and a magnetic quadrupole trap for OH molecules with superposed electric field was specifically addressed. It was concluded that no significant loss enhancement due to electric field would be evident. This is true only for the approximate $^2\Pi_{1/2}$ Hamiltonian used in that study. In~\cite{Stuhl2013} E-fields were applied in our magnetic quadrupole trap to study E-field induced collisions. Although an initial approximation was made of the spin-dynamical effect, subsequent investigations have revealed it to be a threefold underestimate, enough to render deconvolution of any remaining collisional effect difficult. Finally, in~\cite{Bohn2013} it was correctly noted that Hund's case (a) molecules maintain a quantization axis in mixed fields. The states of the molecule were shown to align with one of two quantization axes- either the vector sum or the vector difference of the dipole moment weighted fields $\mu_E\vec{E}$ and $\mu_B\vec{B}$. It was asserted that this would maintain quantization near the zero of a quadrupole trap and avoid spin-flip loss. As we now explain and demonstrate with conclusive experimental evidence, quantization is indeed maintained, but spin-flip loss is enhanced.

%it has taken a concentrated several year effort to elucidate the effect with conclusive experimental evidence as reported here. 

Consider a magnetic quadrupole trap, where a weak-field seeking molecule remains trapped insofar as it adiabatically follows the field direction. Near the trap center, the direction changes most rapidly, causing loss. When electric field is added, it dominates in the trap center where the magnetic field is weakest. Quantization is maintained but the quantization axis does not rotate with the magnetic field. Further away from the trap center the molecule is then magnetically strong field seeking and is lost. To avoid loss, the molecule must switch from the vector sum quantization axis to the vector difference quantization axis, so as to remain magnetically weak field seeking despite the change in relative orientation of the fields. To be more precise, we define the relative orientation of the fields as the sign of $\phi=\vec{E}\cdot\vec{B}$. When $\phi$ is negative (positive), the trapped state must have the vector difference (sum) quantization axis, so that an increase in magnitude of the magnetic field increases its energy. Orientation changes whenever $\phi$ changes sign, which occurs in a 2D region given by $\phi=0$, i.e. \epb. This region must be 2D, since it is a contour level of the 3D scalar valued function $\phi$. 


We can quantify this intuitive picture by diagonalizing the molecular Hamiltonian in mixed fields to find the energy splitting between the well trapped substate and its spin-flip partner. The preceding quantization axis discussion suggests that spin-flips occur when crossing the $\phi=0$ planar region, so we expect to find a correspondingly reduced energy splitting there, since this splitting acts as a barrier to spin-flips. In Fig.~\ref{fig:blocking}, the energies of the well trapped state and its spin-flip partner are calculated by diagonalizing OH's $X^2\Pi_{3/2}$ ground state Hamiltonian verses B-field without E-field  in panel (a), with fixed E-field and $E$\raisebox{1px}{$\parallel$}$B$ so that $\phi$ is maximally nonzero in panel (b), and with fixed E-field and $\phi=0$ in panel (c). Indeed, we find a striking reduction in energy splitting for a wide range of magnetic fields in panel (c) compared with panel (b). In fact, by series expanding the exact eigenenergies of OH, we find $H_\epbm(B)\approx (\mu_BB)^3\Delta^2/(d_EE)^4$, $\Delta$ the lambda doubling term. The Zeeman splitting is no longer linear, but cubic. This means that the splitting will be small in a much larger region close to $B=0$ than otherwise.

%In terms of energies, this manifests as an unusually narrow Zeeman splitting in the region where the relative field orientation changes, i.e. where the fields are orthogonal. In our focus case this occurs in a plane through the trap center, but generally it is always a 2D region since it is a level set of the continuous scalar $\phi = E\cdot B$. In the subset of this plane where $\mu_BB<<d_EE$, the Zeeman effect is not linear but cubic. We call this ``blocking", see fig.~\ref{fig:blocking}, and we say that the E-field blocks the Zeeman effect from linear to cubic. Eventually the Zeeman effect overcomes the blocking and returns to linear when $d_EE\approx\mu_BB$. The Stark effect is not blocked by the Zeeman thanks to lambda doubling, which gives a large fixed energy barrier to electric field misalignment.

\newcommand{\shiftright}[2]{\makebox[#1][r]{\makebox[0pt][l]{#2}}}
\begin{table}[htb]
\caption{Enhancements and loss rates for OH. Evaporation E-field detailed in~\cite{Stuhl2012evap}. Spectroscopic E-field in~\cite{Stuhl2012uwave}. Background loss is $2\text{ s}^{-1}$, experiment length $100\text{ ms}$.}
\label{tab:rates}
\begin{tabular*}{\linewidth}{l*{4}{@{\quad}c}@{\extracolsep{\fill}}l}
\hline\hline
 & \raisebox{-1.3ex}{\shiftright{4pt}{45 mK}} & & \raisebox{-1.3ex}{\shiftright{4pt}{5 mK}} & & \\
\raisebox{1.5ex}{$E$ (V/cm)} & $\eta$ & $\Gamma\,(s^{-1})$ & $\eta$ & $\Gamma\,(s^{-1})$ & \raisebox{1.5ex}{Purpose} \\
\hline
0 		& 1 		& 0.02 	& 1 		& 1.3 	& No Field \\
300 		& 5 		& 0.1 	& 9 		& 11 		& Evaporation \\
550 		& 17 		& 0.3 	& 40 		& 50 		& Spectroscopy \\
3000 	& 1000 	& 19 		& 1600 	& 2000 	& Polarizing \\
\hline\hline
\end{tabular*}
\end{table}

This observation allows us to develop a scaling law for the loss enhancement in a magnetic quadrupole, oriented with strong axis along $\vec{z}$ as in Fig.~\ref{fig:blocking}. For a given trap strength and sample temperature, there is a characteristic energy splitting $\kappa=\sqrt{\hbar\dot{H}}=\sqrt{\hbar dH/dq\sqrt{2k_BT/m}}$ below which spin-flips can occur, calculated from the Landau-Zener formula. In a $2\text{ T/cm}$ gradient with $\mu_B=\text{1.4 bohr}$ and $T=50 \text{mK} as for OH in our previous magnetic quadrupole~\cite{Sawyer2008}, $\kappa=5\text{ MHz}$. As shown in panel (f) of Fig.~\ref{fig:blocking}, E-field widens the $\kappa$ valued energy contour near the trap zero, greatly increasing the flux through this region, which is morphed from a small sphere into a broad thin disk. Note also that the energy gradient near the loss region, which also contributes to the Landau-Zener hopping probability, remains nearly identical in the z-direction between panels (e) and (f). This broad thin disk is responsible for the dramatic enhancement of spin-flip loss relative to atoms. We can calculate an enhancement factor simply by comparing the cross sectional area of the disk with that of the original sphere. Solving for $B$ when $H_\epbm(B)=\kappa$ gives the disk radius, so dividing by the $E=0$ case and squaring gives the flux enhancement factor $\eta = (d_EE/\sqrt{\kappa\Delta})^{8/3}$. Thus E-fields beyond $\sqrt{\kappa\Delta}/d_E$ lead to almost cubic enhancements in spin-flip loss for OH. Rates and enhancements for typical conditions are shown in Table.~\ref{tab:rates}. With the electric field used during evaporation for RF knife purposes in~\cite{Stuhl2012evap}, spin-flip loss is negligible at $50\text{ mK}$ but relevant at $5\text{ mK}$. Thus our new understanding will modify the interpretation of the $5\text{ mK}$ endpoint data, but the data still show enhancements in normalized low-field molecule density at $30\text{ mK}$.


%Those results will thus require reinterpretation considering this effect~\footnote{Spin-flip loss would have interfered with evaporation before $5\text{ mK}$ was attained. Some phase space compression does seem to have occurred at $20\text{ mK}$.}. With the goal of $\mu\text{K}$ temperatures and below, it is clear spin-flip loss must be addressed. 

Generalizing beyond OH, any Hund's case (a) state of any molecule will exhibit a drastically reduced Zeeman splitting near $B=0$ when \epb{} given by: $H_\epbm(B)\approx (\mu_BB)^{2m_J}\Delta^a/(d_EE)^b$, $a$ and $b$ some exponents and $m_J$ the quantum projection number on the total angular momentum $J$. Thus the order of the dependence of the splitting on magnetic field is not always cubic as for OH's $|f,3/2\rangle$ state but is given by $2m_J$. Even states with $m_J=1/2$, which retain a linear Zeeman splitting, are not well trapped for other reasons~\footnote{If $m_J<J$, the state will have avoided crossings with applied $E$, c.f. the $|f,1/2\rangle$ state of OH in gray in panel (c), Fig.~\ref{fig:blocking}, which avoids reduced splitting close to $B=0$ but is un-trapped above $100\text{ G}$ or so. If $m_J=1/2$ there is a negligible g-factor due to cancelation of orbital and spin g-factors.}. This bears out in all test Hamiltonians we have diagonalized, and can be understood intuitively as follows: the Stark effect is blind to the sign of $m_J$, but still sets the quantization axis for $m_J$, so that when the magnetic field is orthogonal to the electric, it sees the eigenfunctions it would normally split with linear efficacy as superpositions of $|m_J\rangle\lm|-m_J\rangle$ instead. These superpositions do not shift at all to first order since the linear shift for each component cancels. The magnetic field eventually splits the levels via perturbation of the eigenbasis itself, a task of higher order according to the degree of polarization, i.e. the magnitude of $m_J$, of the basis state in question.

For Hund's case (b) states the rotation of the molecule, which the electric field interacts with, is not well coupled to the electronic orbit and spin, which the magnetic field interacts with. Thus to first order there is not any competition between quantization axes or any spin-flip loss enhancement. However, no molecular state is perfectly Hund's case (b), they always exhibit some degree of spin-rotation coupling, usually denoted by the parameter $\gamma$. In the energy regime where $\gamma$ is large, the molecule is effectively Hund's case (a) and the spin-flip loss reemerges. $\gamma$ can , for example $\gamma=75\text{ MHz}$ for SrF~\cite{Quemener2016}. In preliminary investigations for Hund's case (b) molecules, which essentially consist of reproducing panels (a)-(c) of Fig.~\ref{fig:blocking} for different Hamiltonians, we find large spin-flip loss enhancements for SrF's v=0, N=1 magnetically trappable substates. Some Hund's case (b) molecular states such as YO's v=0, N=1 manifold have a protected substate with $m_F=0$ and thus no hopping partner in the spin-rotation coupling regime that is nonetheless energetically separated from other state-crossings by the lamb-shift. This state is less strongly trappable due to the same $m_F=0$ feature, but is fully spin-flip immune.

%This informs a general strategy to address spin-flip loss: keep $\mu_BB>\cdot d_EE$ where \epb. 

%The size of the spin-flip loss region can be more precisely computed for different molecules. , but $\mu_BB>\cdot d_EE$ works as an approximate rule. For OH we can analytically express the ground state energy and series expand the result to find $H_\text{Zeeman}\approx (\mu_BB)^3\Delta^2/(d_EE)^4$, $\Delta$ the lambda doubling term. For a general Hund's case (a) molecule, we find the $\mu_BB$ term raised to the power of $2J$. Hund's case (b) suffer to the extent of their non-zero spin-rotation coupling term $\gamma$. 

% Blocking Paragraph:
%Essentially, the loss enhancement is related to the sensitivity of molecules to the relative orientation of $E$ and $B$ fields;  in the familiar atomic case it is only the rotation of the magnetic field relative to the lab frame that induces spin-flips. For Hund's case (a) molecules with full spin-rotation coupling, or for case (b) molecules to the extent that their spin-rotation coupling $\gamma$ is nonzero, electric and magnetic energy shifts add linearly when the fields are parallel but sublinearly when the fields are orthogonal. Consequentially, the presence of a constant orthogonal electric field reduces the magnitude of the Zeeman splitting. We call this effect ``blocking". For OH's most strongly trapped substate, the Zeeman splitting to the next highest state is blocked from linear to cubic as shown in fig.~\ref{fig:blocking}.  The result of this blocking is that when $\mu_BB < d_EE$ and \epb, the energy gap between states of opposite magnetic quantum number is small, and spin-flips can occur. In a magnetic quadrupole trapping geometry with homogeneous overlapping electric field, these conditions are met on a disk through the origin whose size is controlled by the magnitude of $E$. On either side of the disk, blocking returns to linear. This is a worst case scenario, since $P_{\text{flip}}\propto e^{-\Delta^2/(dH/dt)}$ and we have not only small $\Delta$ but large $dH/dt$ for molecules crossing the disk. Panel (e) of fig.~\ref{fig:blocking} illustrates this.



% Splitting Order Paragraph
%To develop some intuition for this, we work in the basis of total angular momentum $J$ and parity $\epsilon$ which is appropriate for Hund's case (a). The eight states in this basis for OH's $J=3/2$ ground state are indicated by the state ket $|\epsilon\!=\!f,e\;;\;m_J\!=\!\pm1/2,\pm3/2\rangle$. An electric field splits states according to the absolute value of their $m_J$ number, with $|f,\pm3/2\rangle$ shifting upward strongly, $|f,\pm1/2\rangle$ one third as strongly, and the negative parity $|e\rangle$ states shifting oppositely. A magnetic field linearly splits states in proportion with $m_J$. Since we use electric fields for slowing and magnetic for trapping, only $|f,3/2\rangle$ are trapped. When an electric field has already set the quantization axis with respect to which $m_J$ is defined, the orthogonal magnetic field sees this as a coupling of $|f,3/2\rangle$ to $|f,-3/2\rangle$ which must be overcome. This is a $3/2-(-3/2)=3^\text{rd}$ order task, hence the cubic Zeeman splitting. If $E\parallel B$, this basis change is unnecessary and the Zeeman splitting is not blocked. For the $|m_J|=1/2$ states, realigning the quantization axis when \epb{} is a $1/2-(-1/2)=1^\text{st}$ order task, so the splitting remains linear; see the gray lines close to zero field in panel (c) of fig.~\ref{fig:blocking}. Similarly, in ref.~\cite{Lara2008}, it was specifically undertaken to investigate the spin-flip loss for OH molecules in a magnetic quadrupole trap with superposed electric field, and no enhancement was found because a simplified $J=1/2$ Hamiltonian was used.


% Relation showing the enhancement
%We can apply Brillouin-Wigner perturbation theory or analytically solve the ground state eigenenergies and taylor expand to obtain the following functional form for the Zeeman splitting between the $|f,\pm3/2\rangle$ states where \epb:
%\begin{equation}
%\label{eq:HZprop}
%H_Z\approx \frac{(1.4\mu_BB)^3}{(1.7d_EE)^4}\Delta^2 f(\Delta,d_EE)
%\end{equation}
%\noindent Here $\mu_B$ and $d_E$ are the bohr magneton and the Debye, $B$ and $E$ are the corresponding field magnitudes, $\Delta$ is the lambda doublet splitting in energy units, and $f$ represents a complicated term of order unity for $d_EE < \Delta$ and order $d_EE/\Delta$ for larger $E$. We can use this to develop a scaling law for the enhancement. Let $\kappa$ be the energy threshold for gaps between states below which spin-flips are possible at the 50\% level. $\kappa$ depends on the mean velocity of trapped species and on the trap gradient near the hopping region, and so must be separately computed for any given scenario. For OH molecules in a  quadrupole trap~\cite{sawyer2008} with strong gradient $2 \text{T/cm}$ and temperature $50 \text{mK}$, $\kappa=5\text{MHz}$. Without electric field, the effective cross sectional area for a hopping region is approximately $\pi (\kappa/\mu_BB\prime)^2$, approximating the ellipsoidal region given by $\mu_BB<\kappa$ as a flat disk. With electric field, a much larger magnetic field is required to overcome blocking, so we have from eq.~\ref{eq:HZprop} that $\mu_BB < \sqrt[3]{\frac{\kappa (d_EE)^4}{\Delta^2}}$, so for $d_EE>\sqrt{\kappa\Delta}$ there is an enhancement given by the following equation:
%\begin{equation}
%\nu = \left(\frac{d_EE}{\sqrt{\kappa\Delta}}\right)^\frac{8}{3}
%\label{eq:blimit}
%\end{equation} 



%~\footnote{For this and other reasons, it is evident that $5\text{ mK}$ temperatures were not actually attained. However it does seem that some phase space compression was achieved for $20\text{ mK}$ evaporations.}. With the goal of much colder temperatures than $5\text{ mK}$, it is clear that the spin-flip loss must be addressed. 

%%%%%%%%%%%%%%%%%%%%%%%%%%
%  PIN TRAP GEOMETRY
%%%%%%%%%%%%%%%%%%%%%%%%%%


%One obvious way to avoid the loss enhancement is to simply never use electric field in a magnetic trap. This prevents loss from being enhanced compared with atoms, but doesn't remove it entirely. Another possibility is to trap with electric fields, where no spin-flip loss is possible thanks to the $\Delta=h\,\cdot\,1.67\text{ GHz}$ splitting between the weak and strong field seeking states. However this splitting also results in a significant reduction in trap gradient close to the center, very undesirable for further cooling by evaporation. Moreover, there are reductions in inelastic to benefit from in magnetic fields.~\cite{Stuhl2012evap}

\begin{figure}[tb]
\includegraphics[width=\linewidth]{Geometry/geometry_out.PNG}%blue-red-yellow-v2_CAD.png}%
\caption{
(a) The trap consists of the last 6 pins of a Stark decelerator. The trap center is directly between the second to last pin pair. These pins have two magnetized domains each. The blue domains are magnetized along $+\hat{y}$, the red along $-\hat{y}$. These pins are grounded, while those in yellow are positively charged. OH is decelerated as in~\cite{Sawyer2008}, except the slowing reaches almost zero velocity at the trap center. (b) Trap energy along axes. $B^\prime=5\text{ T/cm}$ and $E^\prime=100 \text{ kV/cm}^2$. Trap frequencies $\nu_x=3\text{ kHz}$, $\nu_y=5\text{ kHz}$, and $\nu_z=4\text{ kHz}$. Pin-pairs are spaced $2\text{ mm}$, which sets the maximum trap width in the $y$ direction. (c) The electric 2D quadrupole in the $x=0$ plane. The magnetic pins intersect this plane and are shown as black circles. The other pins don't actually intersect this plane, but are projected onto it as yellow rectangles. (d) The magnetic 2D quadrupole in the $z=0$ plane.
\label{fig:CAD}}
\end{figure}
%Detection is realized via LIF with a $282\text{ nm}$ excitation in the x+y-z direction (pink) from $X^2\Pi_{3/2}(v=0)\rightarrow A^2\Sigma(v=1)$, and fluorescence at $313\text{ nm}$ from $A^2\Sigma(v=1)\rightarrow X^2\Pi_{3/2}(v=1)$ is focused by a lens (blue) onto a PMT in the z-direction.

We can generalize to arbitrary geometries and consider methods to avoid the loss using a simple strategy: avoid $\mu_BB < d_EE$ where \epb. One way to achieve this is to trap with E-field and superpose B-field. The lambda doublet prevents flips in this configuration, but it does round the trap minimum considerably. Another option is to trap with both fields and keep zeros overlapped. This was once realized for OH with a superposed magnetic quadrupole and electric hexapole~\cite{Sawyer2007}. Such a scheme prevents spin-flip loss enhancement, but does not remove it entirely. It is also susceptible to misalignment induced loss enhancement. Another possibility is to use one field only. While this avoids spin-flip loss enhancement, any experiment which aims to make use of the doubly dipolar nature of molecules cannot accept this compromise.

%Seeking to remove the loss entirely but without sacrificing trap depth or gradient, we use  
We opt for a geometry that is distinct from these options: a pair of 2D quadrupole traps, one magnetic and the other electric, with orthogonal centerlines. We achieve these fields in a geometry that matches our Stark decelerator~\cite{Bochinski2003}, as shown in Fig.~\ref{fig:CAD}. This approach is similar to the Ioffe-Pritchard strategy~\cite{pritchard1983}, where a 2D magnetic quadrupole is combined with an axial dipole trap. Axial and radial trapping interfere, resulting in significantly lower trap depths than the 3D quadrupole. We thwart this interference by using electric field for the third direction. Our geometry has \epb{} along both the $xz$ and $yz$ planes, with $\mu_BB < d_EE$ in a large cylinder surrounding the $z$-axis. However, by adding magnetic field $\vec{B}=B_\text{coil}\hat{z}$ along the centerline of the magnetic quadrupole with an external bias coil, a fully tunable scenario emerges. %the \epb{} surface can be morphed into a hyperbolic sheet that does not approach closely to the magnetic minimum axis except far away from the trap center, thus preventing $\mu_BB < d_EE$.% , which can ``block" one another as discussed earlier but never result in an absolute decrease in potential energy of the doubly weak field seeking substate. 

%According to our intuition and our phase space coupling simulations, this integrated trap represents a near best-case scenario for coupling between a pulsed decelerator and a trap and ought to provide trap-averaged temperatures below $100\text{ mK}$. In practice we do not realize any significant molecule number increase relative to our previous trap geometries and temperatures are $170\text{ mK}$. This could be related to the difficulty of conditioning the magnet surfaces, which could be micro-discharging during operation leading to heating during loading. Finer polishing of the magnetic pins should address this.

\begin{figure}[tb]
\includegraphics[width=\linewidth]{LossSurfaces/losssurfaces_out.png}%
\caption{
Surfaces where spin-flips can occur for several values of $B_\text{coil}$. These surfaces are subsets of those where \epb{}, shown only where $\mu_BB$ is small enough relative to $d_EE$ to allow spin-flips.
\label{fig:LSurfs}}
\end{figure}

Adding $B_\text{coil}$ only slightly rounds the magnetic trapping potential, but it morphs the \epb{} surface from a pair of planes into a hyperbolic sheet, pushing it away from the $z$-axis where the magnetic field is smallest. This means that small magnitudes of $B_\text{coil}$  are sufficient to avoid $\mu_BB< d_EE$ where \epb. In Fig.~\ref{fig:LSurfs}, the surfaces where \epb{} for several $B_\text{coil}$ magnitudes are shown wherever the splitting there is below the hopping threshold $\kappa$. Note how $B_\text{coil}$ tunes the proximity of the loss regions to the trap center. The loss regions are always visible, but they are tuned so far away that molecules accessing them have already escaped the trap. The striking difference in molecule trap lifetime with and without $B_\text{coil}$ can be seen in Fig.~\ref{fig:WVplot}, panel (a).

\begin{figure}[tb]
\includegraphics[width=\linewidth]{VWFig/vwfig_out.png}%
\caption{
(a) Time traces without bias field (black), with bias field (green dots), and with modulated density (green circles). A one body fit (gray) to the data without bias field yields a $200\text{ s}^{-1}$ loss rate. A one body fit (black) to the long time bias field data yields a $2\text{ s}^{-1}$ loss rate, in agreement with our background gas pressure. (b) At the fixed time $30 \text{ms}$, population is shown as a function of both pin translation and bias field.
\label{fig:WVplot}}
\end{figure}

%allows a full tuning from $200\text{ s}^{-1}$ to complete removal, see 

As a further confirmation of our \epb{}  and $\mu_BB<d_EE$ model of the loss, we translate our magnetic pins along the $\hat{x}$ direction in their mounts to alter the surface where \epb{}  and compare experimental data against our expectations. The data are shown in Fig.~\ref{fig:WVplot}, panel (b). Qualitatively, this translation serves to disrupt the otherwise perfectly 2D magnetic quadrupole by adding a small trapping field $\vec{B}\propto B^\prime z\hat{z}$ along the z-axis. This means that $B_\text{coil}$ no longer directly tunes the magnetic field magnitude along the z-axis. Instead, $B_\text{coil}$ must first overcome the slight trapping field along the $z$-axis, translating a point of zero field along the z axis and eventually out of the trap. The point of zero magnetic field has $\mu_BB<d_EE$ and lies on the $\phi=\vec{E}\cdot\vec{B}=0$ surface by definition, leading to strong loss unless it is aligned with the trap center, where $E$ is also zero. This means that without any bias field, the loss should actually be a local minimum; as the field is increased in either direction the loss should first worsen and then improve when the zero leaves the trap. This qualitative explanation correctly predicts the observed double well structure.

Quantitatively, we fit the family of curves shown in Fig.~\ref{fig:WVplot}(b) by performing an integration of molecule flux weighted by Landau-Zener probability and Maxwell-Boltzmann population density over the strangely twisted surfaces where \epb{} for each $B_\text{coil}$ and pin offset. The computation is performed in COMSOL Multiphysics, with cloud temperature as the only free parameter~\footnote{Soure code: https://github.com/dreens/spin-flip-integration/}. The asymmetry of the curves about $B_\text{coil}=0$ comes from a slight shift of the electric quadrupole minimum caused by an intentional bending of the last pin pair to increase fluorescence collection along $\hat{z}$. The fitted temperature is in the $100-200$ mK range, larger than expected from our simulations, despite the known defocusing and reflection losses that accompany pulsed decelerators at low speeds~\cite{Sawyer2008a}. This may be related to micro-discharges on the surfaces of the magnetic pins during the final deceleration pulse, since these pins are less well polished. This could be overcome with various polishing strategies.

\begin{figure}[tb]
\includegraphics[width=\linewidth]{MWSpec/mwspec_out.png}%
\caption{
Microwave Thermometry at different values of $B_\text{coil}$. Increasing $B_\text{coil}$ increases and shifts population as expected. The location of the lowest trap energy at which a molecule can access loss regions for a given $B_\text{coil}$ is indicated with a correspondingly colored triangle on the magnetic field axis.
\label{fig:spec}}
\end{figure}

We further confirm the consistency of our model of the loss using microwave spectroscopy as performed as in our previous work~\cite{Stuhl2012evap}. Instead of using a bias tee setup, we use a microwave probe to excite free space modes of our vacuum chamber in a near field manner. The results are shown in Fig.~\ref{fig:spec}. It is seen that increasing $B_{\text{coil}}$ increases population first at low fields and then at higher fields. This is consistent with our calculations of the location of the loss as presented in Fig.~\ref{fig:LSurfs}. In order to perform this spectroscopy, the trapping electric fields are switched off immediately prior to Zeeman specific microwave transfer pulses. Thus the results reflect the Zeeman potential energy only, which in the ensemble average tracks potential energy but is not directly proportional. Nonetheless, the shift in population center is clear and in agreement with our expectation.  %Roughly speaking, the average field of $2\text{ kG}$ corresponds to $200\text{ mK}$ for OH. Since this is approximately half the potential energy, we have $U\approx400\text{ mK}$. From the virial theorem for a linear trap, $U = 4.5k_BT$, so we can say the spectrum is consistent with $T\approx 90\text{ mK}$ and consistent with our fitting in fig.~\ref{fig:WVplot}.

%a challenging prospect with our trap deeply integrated in the high voltage decelerator, 

In the case of lowest applied magnetic field in Fig.~\ref{fig:spec}, a negative going signal is observed. This indicates a build-up in the opposite parity weak electric field seeking state. Although the spin-flips we have discussed do not connect states of opposite parity, other avoided crossings amongst the ground state manifold result in the spin-flipped molecules remaining very weakly trapped in a secondary state with opposite parity character in some regions. %nonuniform gradient (repulsive in the center, attractive outside) and a much lower overall trap depth. This secondary state also exhibits spin-flip loss to other lower states, but the enhancement with electric field is related to a quadratic blocking of the Zeeman splitting, and is thus not as dominating as the loss in the primary state due to cubic blocking.

With loss removed, we observe the green-dotted trend in Fig.~\ref{fig:WVplot}(a). The decay rate decreases with population over the first $200\text{ ms}$, a timescale that is long compared with trap frequency and thus suggestive of a collisional process. To test this, we reduce the molecule number fivefold with a technique that ought to be phase-space blind so as to reduce the molecule density. We do this with microwaves as before, but coupling electrically weak and strong field seekers during the first half of deceleration. Spatial inhomogeneities on the order of the cloud size are unlikely given the $15\text{ cm}$ microwave wavelength, but any that exist are remixed during the remaining deceleration. The trend remains similar with this density-modulation, suggesting that single-particle physics is chiefly responsible. This discrepancy relative to our previous work is attributable to our warmer initial temperature. We hope soon to implement several density increasing measures and return to the collisional regime. The slowly varying but single-body decay rate could be related to the existence of high energy chaotic orbits with long escape times, as seen in other exotically shaped trapping potentials~\cite{Gonzalez-Ferez2014}.

Our dual quadrupole trap decisively overcomes molecule enhanced spin-flip loss by tuning it from an overwhelming rate to complete removal. Our explanation of the loss provides detailed predictions of how its location and magnitude ought to scale with bias field and trap alignment, which we have experimentally verified. Our results correct existing predictions about molecular spin-flips in mixed fields and we provide a consistent framework that explains this based on internal spin-dynamics. We have devised a viable trapping geometry in which spin-flip loss is fully mitigated without trap-depth sacrifice, paving the way toward further improvements in molecule trapping and cooling.

We acknowledge the Gordon and Betty Moore Foundation, the ARO-MURI, JILA PFC, and NIST for their financial support. T.L. acknowledges support by the Alexander von Humboldt Foundation through a Feodor Lynen Fellowship. We thank J.L. Bohn and S.Y.T. van de Meerakker for helpful discussions. We thank Goulven Qu\'em\'ener for his continued involvement.

D.R. and H.W. contributed equally: D.R. in writing and trap design, H.W. in experiment execution.

%includes uncited bib entries
%\nocite{*}
\bibliographystyle{apsrev4-1_no_Arxiv}
\bibliography{MolecularSpinFlipLoss}% Produces the bibliography via BibTeX.

\end{document}
%
% ****** End of file MolecularMajoranaLoss.tex ******
