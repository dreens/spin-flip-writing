	% ****** Start of file MolecularSpinFlipLoss.tex ******
%
%
%

\documentclass[%
 reprint,
%superscriptaddress,
%groupedaddress,
%unsortedaddress,
%runinaddress,
%frontmatterverbose, 
%preprint,
%showpacs,preprintnumbers,
%nofootinbib,
%nobibnotes,
%bibnotes,
 amsmath,amssymb,
 aps,
prl,
%pra,
%prb,
%rmp,
%prstab,
%prstper,
%floatfix,
]{revtex4-1}

\usepackage{graphicx}% Include figure files
\usepackage{dcolumn}% Align table columns on decimal point
\usepackage{bm}% bold math
\usepackage[hidelinks]{hyperref}% add hypertext capabilities
%\usepackage[mathlines]{lineno}% Enable numbering of text and display math
%\linenumbers\relax % Commence numbering lines
\usepackage{textcomp}

%\usepackage[showframe,%Uncomment any one of the following lines to test 
%%scale=0.7, marginratio={1:1, 2:3}, ignoreall,% default settings
%%text={7in,10in},centering,
%%margin=1.5in,
%%total={6.5in,8.75in}, top=1.2in, left=0.9in, includefoot,
%%height=10in,a5paper,hmargin={3cm,0.8in},
%]{geometry}


\newcommand{\bcl}{{$B_\text{coil}$}}
\newcommand{\epb}{{$E\!\perp\!B$}}
\newcommand{\epbm}{{E\!\perp\!B}}
\newcommand{\cmnt}[1]{\ignorespaces}



\begin{document}

%\preprint{APS/123-QED}

\title{Molecular Spin-Flip Loss and a Dual Quadrupole Trap}%

\author{David Reens}
\thanks{Contributed equally. Email dave.reens@colorado.edu or hao.wu@colorado.edu.}
%\affiliation{JILA, National Institute of Standards and Technology and the University of Colorado}
%\affiliation{Department of Physics, University of Colorado, Boulder, Colorado 80309-0440, USA}
\author{Hao Wu}
\thanks{Contributed equally. Email dave.reens@colorado.edu or hao.wu@colorado.edu.}
%\affiliation{JILA, National Institute of Standards and Technology and the University of Colorado}
%\affiliation{Department of Physics, University of Colorado, Boulder, Colorado 80309-0440, USA}
\author{Tim Langen}%
\altaffiliation{Present Address: 5. Physikalisches Institut and Center for Integrated Quantum Science and Technology (IQST), Universit\"at Stuttgart, Pfaffenwaldring 57, 70569 Stuttgart, Germany}
%\affiliation{JILA, National Institute of Standards and Technology and the University of Colorado}
\author{Jun Ye}
\affiliation{JILA, National Institute of Standards and Technology and the University of Colorado and\\ Department of Physics, University of Colorado, Boulder, Colorado 80309-0440, USA}


\date{\today}% It is always \today, today,


%%%%%%%%%%%%%%%%%%%%%
%ABSTRACT
%%%%%%%%%%%%%%%%%%%%%
\begin{abstract}
Doubly dipolar molecules exhibit complex internal spin-dynamics when electric and magnetic fields are both applied. Near magnetic trap minima, these spin-dynamics lead to enhancements in Majorana spin-flip transitions by many orders of magnitude relative to atoms, and are thus an important obstacle for progress in molecule trapping and cooling. The effect is strongest for Hund's case (a) states and is significant for Hund's case (b) as well. We study these internal spin-dynamics with OH molecules and devise a trap geometry where spin-flip loss can be tuned from over $200 \text{ s}^{-1} $ to below our $2\text{ s}^{-1}$ vacuum limited loss rate with only a simple external bias coil and with no sacrifice of trap strength.
%A new electromagnetic trap geometry allows full tuning of complex molecular spin-dynamics in crossed electric and magnetic fields. If not tuned properly, these dynamics lead to spin-flip loss that afflicts a wide set of candidate molecules. The spin-flip loss can be significant even above $100\text{ mK}$ and varies inversely with temperature, so its removal represents a critical step toward ultracold molecules. The trapping geometry features a $0.5 \text{ K}$ trap depth and $5 \text{ T/cm}$ trap strength, and allows spin-dynamics to be tuned with a weak external bias coil. Spin-flip loss is tuned in a $170 \text{ mK}$ sample of OH molecules from over $200 \text{ s}^{-1} $ to below the vacuum limited lifetime of $2 \text{ s}^{-1}$.
\end{abstract}


\maketitle


%%%%%%%%%%%%%%%%%%%%%%%%%%%%%%%%%
%
%     III   NNN   TTT   RRR   OOO   DDD   UUU   CCC   TTT   III   OOO   NNN
%     III   NNN   TTT   RRR   OOO   DDD   UUU   CCC   TTT   III   OOO   NNN
%     III   NNN   TTT   RRR   OOO   DDD   UUU   CCC   TTT   III   OOO   NNN
%
%%%%%%%%%%%%%%%%%%%%%%%%%%%%%%%%%
%\section{Introduction}
The ultracold regime extends toward molecules on many fronts~\cite{Carr2009}. KRb molecules have reached lattice quantum degeneracy~\cite{Moses2015} and other bialkalis continue to progress~\cite{Takekoshi2014, Park2015,Guo2016,Liu2017}. Creative and carefully engineered laser cooling strategies are tackling certain nearly vibrationally diagonal molecules~\cite{Stuhl2008,Shuman2010,Hummon2013, Barry2014, Zhelyazkova2014, Steinecker2016, Hemmerling2016}. A diverse array of alternative strategies have succeeded to greater or lesser extents on other molecules~\cite{Doyle1998, Bethlem1999, Bochinski2003, Narevicius2008, Wiederkehr2012, Prehn2016,Liu2017a}. All of these molecules will require secondary strategies like evaporation or sympathetic cooling to make further gains in phase space density~\cite{Parazzoli2011, Stuhl2012evap, Quemener2016}. They also may face a familiar challenge: spin flip loss near the zero of a magnetic trap, but dramatically enhanced for many doubly dipolar molecules due to their internal spin dynamics in mixed electric and magnetic fields. 

The knowledge of spin flips or Majorana hops as an eventual trap lifetime limit predates the very first magnetic trapping of neutrals~\cite{Migdall1985}. Spin flips were directly observed near $50\,\mu\text{K}$ and overcome with a time-orbiting potential trap~\cite{Petrich1995} and a plugged dipole trap~\cite{Davis1995}, famously enabling the first production of Bose-Einstein condensates. In our earlier investigations, we observed loss of magnetically trapped hydroxyl radicals (OH) with applied electric field~\cite{Stuhl2012uwave}. This trap loss occurred for sub-states of OH's $\mathrm{X}^2\Pi_{J=3/2}$ ground state manifold other than the most well trapped one (positive parity and full spin polarization, $|f,m_J=3/2\rangle$, blue in Fig.~\ref{fig:blocking}). Due to the closely spaced parity doublet, a general feature of Hund's case (a), these states intersect with opposite parity states at non-zero magnetic fields, where electric field can open avoided crossings and cause trap loss. We now identify internal spin-dynamics leading to trap loss near zero B-field even for the most well trapped state and even at $50\text{ mK}$.

%We first suspected this during experiments in our previous trap geometry, a 3D permanent magnet quadrupole trap with homogeneous electric field~\cite{Stuhl2013}, but it was difficult to disentangle it from competing effects.  Our new trap geometry addresses this, but we use our previous geometry as our starting point to explain the internal spin-dynamics that lead to the enhanced loss.

\begin{figure}[tb]
\includegraphics[width=\linewidth]{Blocking/blocking_out.png}%
\caption{
Four Zeeman split lines in OH's $\mathrm{X}^2\Pi_{3/2}$ manifold are shown (a-c), with the trapped $|f,3/2\rangle$ state in blue and its spin-flip partner $|f,-3/2\rangle$ in dashed red. These states are shown with no E-field (a), with $E=150\text{ V/cm}$ and $E\,$\raisebox{0.5px}{$\parallel$}$\,B$ (b), and with \epb{} (c). Note the vastly reduced red-blue splitting in the latter case. The opposite parity ($|e\rangle$) manifold of strong E-field seekers is split by $\Delta$ (d). Energy splitting contours are shown every $2\text{ mK}$ near the zero of a $2\text{ T/cm}$ magnetic quadrupole trap for OH molecules~\cite{Stuhl2012uwave} without E-field (e), and with uniform $E=150\text{ V/cm}$ along the strong axis of the quadruploe (f). The vectors give the quantization axis, which for well trapped molecules is $\mu_BB-d_EE$ above the centerline and $\mu_BB+d_EE$ below. Note the drastic widening of the lowest energy splitting contour, the culprit for drastic molecular spin-flip loss enhancement here discussed.
\label{fig:blocking}}
\end{figure}
%B-field vectors in blue. (f) Again with $E=150\text{ V/cm}$. Note drastic widening of lowest contour (red). Vector direction gives the effective quantization axis of the trapped state, $\mu_BB\pm d_EE$ above (below) the horizontal centerline. Vector magnitude gives potential energy relative to trap center.


These internal spin-dynamics are subtle, having eluded three previous investigations of note: In~\cite{Lara2008} the analogues of atomic spin-flip loss for molecules in mixed fields were modeled, and a magnetic quadrupole trap for OH molecules with superposed electric field was specifically addressed. It was concluded that no significant loss enhancement due to electric field would be evident. This is true only for the approximate $^2\Pi_{1/2}$ Hamiltonian used in that study. In~\cite{Stuhl2013} E-fields were applied in our magnetic quadrupole trap to study E-field induced collisions. Although an initial approximation was made of the spin-dynamical effect, subsequent investigations have revealed it to be a threefold underestimate, enough to render deconvolution of any remaining collisional effect difficult. Finally, in~\cite{Bohn2013} it was correctly noted that Hund's case (a) molecules maintain a quantization axis in mixed fields. The states of the molecule were shown to align with one of the two quantization axes given by $\mu_E\vec{E}\pm\mu_B\vec{B}$. It was asserted that this would maintain quantization near the zero of a quadrupole trap and avoid spin-flip loss. As we now explain and demonstrate with conclusive experimental evidence, electric field does maintain quantization, but spin-flip loss is enhanced.

%it has taken a concentrated several year effort to elucidate the effect with conclusive experimental evidence as reported here. 

Consider a magnetic quadrupole trap, where a weak-field seeking molecule remains trapped insofar as it adiabatically follows the field direction. Near the trap center, the direction changes most rapidly, causing spin-flips. When electric field is added, it dominates in the trap center where the magnetic field is weakest. Quantization is maintained but the quantization axis does not rotate with the magnetic field. Further away from the trap center the molecule is then magnetically strong field seeking and is lost. To avoid loss, the molecule must switch from the vector sum quantization axis to the vector difference quantization axis, so as to remain magnetically weak field seeking despite the change in relative orientation of the fields. To be more precise, we define the relative orientation of the fields as the sign of $\phi=\vec{E}\cdot\vec{B}$. When $\phi$ is negative (positive), the trapped state must have the vector difference (sum) quantization axis, so that an increase in magnitude of the magnetic field increases its energy. Orientation changes whenever $\phi$ changes sign, i.e. where $\phi=0$ which means \epb. This happens in a 2D region, since $\phi=0$ is a contour level of the 3D scalar valued function $\phi$. 


We can quantify this intuitive picture by diagonalizing the molecular Hamiltonian in mixed fields to find the energy splitting between the well trapped substate and its spin-flip partner (Fig.~\ref{fig:blocking}a-c). The preceding quantization axis discussion suggests that spin-flips occur when crossing the $\phi=0$ planar region, so we expect to find a correspondingly reduced energy splitting there, since this splitting acts as a barrier to spin-flips. This is indeed the case; compare panels (b-c). In fact, by series expanding the exact eigenenergies of OH, we find $H_\epbm(B)\approx (\mu_BB)^3\Delta^2/(d_EE)^4$, $\Delta$ the lambda doubling term. The Zeeman splitting is no longer linear, but cubic. This means that the splitting will be small in a much larger region than otherwise.

%In terms of energies, this manifests as an unusually narrow Zeeman splitting in the region where the relative field orientation changes, i.e. where the fields are orthogonal. In our focus case this occurs in a plane through the trap center, but generally it is always a 2D region since it is a level set of the continuous scalar $\phi = E\cdot B$. In the subset of this plane where $\mu_BB<<d_EE$, the Zeeman effect is not linear but cubic. We call this ``blocking", see fig.~\ref{fig:blocking}, and we say that the E-field blocks the Zeeman effect from linear to cubic. Eventually the Zeeman effect overcomes the blocking and returns to linear when $d_EE\approx\mu_BB$. The Stark effect is not blocked by the Zeeman thanks to lambda doubling, which gives a large fixed energy barrier to electric field misalignment.

This observation allows us to develop a scaling law for the loss enhancement in a magnetic quadrupole trap. The Landau-Zener formula~\cite{Rubbmark1981} allows calculation of a characteristic energy splitting $\kappa$ for spin-flips. Molecules whose energy approaches that of another state more closely than $\kappa$ will flip, depending also on the rate of approach of the energy levels. In a quadrupole trap the rate of approach is the product of molecule velocity and trap gradient. Assuming the thermal average velocity, $\kappa$ can be calculated as $5\text{ MHz}$ for $50\text{ mK}$ OH in our previous magnetic quadrupole~\cite{Sawyer2008}. The spin-flip loss rate can be derived from $\kappa$ by computing the molecule flux into the ellipsoid where the energy splitting is below $\kappa$. E-field widens this ellipsoid into a broad, thin disk~\footnote{Or an ellipse if $E$ points along a different axis; this influences the scaling law by at most a factor of two.}  (Fig.~\ref{fig:blocking}e-f), whose radius is found by solving for $B$ when $H_\epbm(B)=\kappa$. Dividing by the original radius and squaring gives the enhancement in molecule flux through this disk-like loss region, the factor $\eta = (d_EE/\sqrt{\kappa\Delta})^{8/3}$. Thus, E-fields beyond $\sqrt{\kappa\Delta}/d_E$ lead to almost cubic enhancements in spin-flip loss for OH (Tab.~\ref{tab:rates}). This new understanding modifies our interpretation of evaporation data for OH~\cite{Stuhl2012evap}, especially at $5\text{ mK}$ where the loss rate is significantly enhanced by the $E$-fields used for RF knife purposes. We do still find enhancements in normalized low-field density for shallow RF knife cuts from $55-30\text{ mK}$.


\newcommand{\shiftright}[2]{\makebox[#1][r]{\makebox[0pt][l]{#2}}}
\begin{table}[t]
\caption{Enhancements and loss rates for OH with typical applied fields. Zero field values are equivalent to atomic spin-flip loss. E-field is required during evaporation and spectroscopy to open avoided crossings for $|e\rangle$ parity states~\cite{Stuhl2012evap,Stuhl2012uwave}. Background loss is $2\text{ s}^{-1}$, experiment length $100\text{ ms}$.}
\label{tab:rates}
\begin{tabular*}{\linewidth}{l*{4}{@{\quad}c}@{\extracolsep{\fill}}l}
\hline\hline
 & \raisebox{-1.3ex}{\shiftright{4pt}{55 mK}} & & \raisebox{-1.3ex}{\shiftright{4pt}{5 mK}} & & \\
\raisebox{1.5ex}{$E$ (V/cm)} & $\eta$ & $\Gamma\,(s^{-1})$ & $\eta$ & $\Gamma\,(s^{-1})$ & \raisebox{1.5ex}{Purpose} \\
\hline
0 		&1 		& 0.02 	& 1 		& 1.3 	& Zero Field \\
300 		& 5 		& 0.1 	& 9 		& 11 		& Evaporation \\
550 		& 17 		& 0.3 	& 40 		& 50 		& Spectroscopy \\
3000 	& 1000 	& 19 		& 1600 	& 2000 	& Polarizing \\
\hline\hline
\end{tabular*}
\end{table}

%As shown in panel (f) of Fig.~\ref{fig:blocking}, E-field widens the $\kappa$ valued energy contour near the trap zero, greatly increasing the flux through this region, which morphs from a small sphere into a broad thin disk. Note also that the energy gradient near the loss region, which also contributes to the Landau-Zener hopping probability, remains nearly identical in the z-direction between panels (e) and (f). This broad thin disk is responsible for the dramatic enhancement of spin-flip loss relative to atoms. We can calculate an enhancement factor simply by comparing the cross sectional area of the disk with that of the original sphere. Solving for $B$ when $H_\epbm(B)=\kappa$ gives the disk radius, so dividing by the $E=0$ case and squaring gives the flux enhancement factor $\eta = (d_EE/\sqrt{\kappa\Delta})^{8/3}$. Thus E-fields beyond $\sqrt{\kappa\Delta}/d_E$ lead to almost cubic enhancements in spin-flip loss for OH. Rates and enhancements for typical conditions are shown in Table.~\ref{tab:rates}. With the electric field used during evaporation for RF knife purposes in~\cite{Stuhl2012evap}, spin-flip loss is negligible at $50\text{ mK}$ but relevant at $5\text{ mK}$. Thus our new understanding will modify the interpretation of the $5\text{ mK}$ endpoint data, but the data still show enhancements in normalized low-field density at $30\text{ mK}$.

%, oriented with strong axis along $\vec{z}$ as in Fig.~\ref{fig:blocking}e-f. 

%Those results will thus require reinterpretation considering this effect~\footnote{Spin-flip loss would have interfered with evaporation before $5\text{ mK}$ was attained. Some phase space compression does seem to have occurred at $20\text{ mK}$.}. With the goal of $\mu\text{K}$ temperatures and below, it is clear spin-flip loss must be addressed. 

Generalizing beyond OH, Hund's case (a) states exhibit reduced Zeeman splittings near $B\!=\!0$ when \epb{} that are not always cubic as for OH but satisfy $H_\epbm(B)\propto (\mu_BB)^{2m_J}$. Only states with $m_J=1/2$ retain a linear Zeeman splitting, but they are either not well trapped if $J=1/2$ also, or they are initially linear until avoided crossings with other states (c.f. $|f,1/2\rangle$ for OH, gray close dashes, Fig.~\ref{fig:blocking}c). This bears out in all test Hamiltonians we have diagonalized, and can be understood intuitively as follows: the Zeeman effect is a perturbation on a Hamiltonian whose quantization axis, which defines the $|m_J\rangle$ quantum number, has already been set by the Stark effect. Normally the Zeeman effect would linearly split states according to $|m_J\rangle$, but from its perspective the states are superpositions $|m_J\rangle\pm|\!-\!m_J\rangle$ and do not split at all to first order. Only via perturbations of the eigenbasis itself does a Zeeman splitting emerge.


%the Stark effect is blind to the sign of $m_J$, but still sets the quantization axis for $m_J$, so that when the magnetic field is orthogonal to the electric, it sees the eigenfunctions it would normally split with linear efficacy as superpositions of $|m_J\rangle\pm |-m_J\rangle$ instead. These superpositions do not shift at all to first order since the linear shift for each component cancels. The magnetic field eventually splits the levels via perturbation of the eigenbasis itself, a task of higher order according to the degree of polarization, i.e. the magnitude of $m_J$, of the basis state in question.

%

Generalizing beyond Hund's case (a), any state which exhibits competition between the magnetic and electric fields for the alignment of the molecule will be susceptible to this effect. One way to avoid this competition is for the fields to couple to unrelated parts of the Hamiltonian, which does happen to a limited extent for Hund's case (b) states without electron orbital angular momentum ($\Sigma$ states, $\Lambda=0$)~\cite{Bohn2013}. In these states, which include most laser-cooled and bialkali molecules thus far, the electric and magnetic fields couple to rotation and spin respectively, which are only related by the spin-rotation coupling constant $\gamma$. Where the fields are large compared with $\gamma$, they stop competing, and both lead to linear energy shifts regardless of the magnitude of the other. This means that the cross sectional area of loss regions is enhanced to the extent that $\kappa<\gamma$ but is then bounded. Typical molecules have $\gamma$ in the tens of MHz~\cite{Quemener2016}, and $\kappa$ much lower than the $5\text{ MHz}$ for our warm and very tightly confined OH; so spin-flip loss enhancement remains important. In some cases hyperfine interactions can subvert the loss enhancement, which we calculate for some states of yttrium oxide (YO)~\footnote{This is exciting given the very recently realized 3D MOT for YO. We intend to publish more regarding spin-flip immune states for YO and other molecules.}. %This state avoids spin-flips by having $m_F=0$ in the large $\gamma$ regime, where $F$ is the total angular momentum including nuclear spin playing the role that $J$ did in our previous explanations. It has no spin-flip partner since $-m_F=m_F$, but is still separated above other states due to the fermi contact interaction. %such as the ground vibrational first rotational state of yttrium oxide (YO) 


%%%%%%%%%%%%%%%%%%%%%%%%%%
%  PIN TRAP GEOMETRY
%%%%%%%%%%%%%%%%%%%%%%%%%%



\begin{figure}[tb]
\includegraphics[width=\linewidth]{Geometry/geometry_out.PNG}%blue-red-yellow-v2_CAD.png}%
\caption{
The last six pins of our Stark decelerator~\cite{Sawyer2008} form the trap (a), which is $0.45\text{ mK}$ deep with trap frequency $\nu\approx4\text{ kHz}$ (b). Along $y$ the trap is bounded by the $2\text{ mm}$ pin spacing. The yellow pins are positively charged and the central pin pair negatively, which forms a 2D electric quadrupole trap with zero along the $x$-axis. This is shown for the $x=0$ plane (c), with yellow pins artificially projected for clarity since they don't actually intersect the plane. The central pins are magnetized, with two domains each. Blue indicates magnetization along $+\hat{y}$, red along $-\hat{y}$. These domains produce a magnetic quadrupole trap with zero along the $z$-axis, shown in the $z=0$ plane (d). %(a) The trap consists of the last 6 pins of a Stark decelerator. The trap center is directly between the second to last pin pair. These pins have two magnetized domains each. The blue domains are magnetized along $+\hat{y}$, the red along $-\hat{y}$. These pins are grounded, while those in yellow are positively charged. OH is decelerated as in~\cite{Sawyer2008}, except the slowing reaches almost zero velocity at the trap center. (b) Trap energy along axes. $B^\prime=5\text{ T/cm}$ and $E^\prime=100 \text{ kV/cm}^2$. Trap frequencies $\nu_x=3\text{ kHz}$, $\nu_y=5\text{ kHz}$, and $\nu_z=4\text{ kHz}$. Pin-pairs are spaced $2\text{ mm}$, which sets the maximum trap width in the $y$ direction. (c) The electric 2D quadrupole in the $x=0$ plane. The magnetic pins intersect this plane and are shown as black circles. The other pins don't actually intersect this plane, but are projected onto it as yellow rectangles. (d) The magnetic 2D quadrupole in the $z=0$ plane.
\label{fig:CAD}}
\end{figure}
%Detection is realized via LIF with a $282\text{ nm}$ excitation in the x+y-z direction (pink) from $X^2\Pi_{3/2}(v=0)\rightarrow A^2\Sigma(v=1)$, and fluorescence at $313\text{ nm}$ from $A^2\Sigma(v=1)\rightarrow X^2\Pi_{3/2}(v=1)$ is focused by a lens (blue) onto a PMT in the z-direction.

We can generalize to arbitrary geometries and consider methods to suppress the loss using a simple strategy: avoid $\mu_BB < d_EE$ where \epb. One way to achieve this is to trap with E-field and superpose B-field. The lambda doublet prevents flips in this configuration, but it correspondingly rounds the trap minimum, weakening confinement. Another option is to trap with both fields and keep zeros overlapped. This was once realized for OH with a superposed magnetic quadrupole and electric hexapole~\cite{Sawyer2007}. Such a scheme prevents spin-flip loss enhancement, but does not remove it entirely. It is also susceptible to misalignment induced loss enhancement. Another possibility is to use one field only, but any experiment which aims to make use of the doubly dipolar nature of molecules cannot accept this compromise. 

We opt for a new geometry: a pair of 2D quadrupole traps, one magnetic and the other electric, with orthogonal centerlines (Fig.~\ref{fig:CAD}). We achieve these fields in a geometry that matches our Stark decelerator~\cite{Bochinski2003}. This approach is similar to the Ioffe-Pritchard strategy~\cite{pritchard1983}, where a 2D magnetic quadrupole is combined with an axial magnetic dipole trap. While this successfully prevents spin-flip loss, axial and radial trapping interfere, resulting in significantly lower trap depths than the 3D quadrupole. We thwart this interference by using electric field for the third direction. Our geometry has \epb{} in the two planes $x\cdot y=0$, and $\mu_BB < d_EE$ in a large cylinder surrounding the $z$-axis. However, by adding magnetic field $\vec{B}=B_\text{coil}\hat{z}$ along the centerline of the magnetic quadrupole with an external bias coil, a fully tunable scenario emerges. %the \epb{} surface can be morphed into a hyperbolic sheet that does not approach closely to the magnetic minimum axis except far away from the trap center, thus preventing $\mu_BB < d_EE$.% , which can ``block" one another as discussed earlier but never result in an absolute decrease in potential energy of the doubly weak field seeking substate. 

%According to our intuition and our phase space coupling simulations, this integrated trap represents a near best-case scenario for coupling between a pulsed decelerator and a trap and ought to provide trap-averaged temperatures below $100\text{ mK}$. In practice we do not realize any significant molecule number increase relative to our previous trap geometries and temperatures are $170\text{ mK}$. This could be related to the difficulty of conditioning the magnet surfaces, which could be micro-discharging during operation leading to heating during loading. Finer polishing of the magnetic pins should address this.

\begin{figure}[tb]
\includegraphics[width=\linewidth]{LossSurfaces/losssurfaces_out.png}%
\caption{
Surfaces where spin-flips can occur (\epb{}, $\mu_BB<d_EE$) are shown for three values of \bcl{} in light gray, dark gray, and black. The magnetic pins are shown as in Fig.~\ref{fig:CAD} for context. The purple star marks the trap center, to which molecules are confined within a \raisebox{2.5px}{\texttildelow} $\!\!1\text{ mm}$ diameter.
\label{fig:LSurfs}}
\end{figure}

Adding \bcl{} only slightly rounds the magnetic trapping potential, but it morphs the \epb{} surface from a pair of planes into a hyperbolic sheet ($x\cdot y\propto B_\text{coil}\cdot z$), pushing it away from the $z$-axis where the magnetic field is smallest. Thus, small magnitudes of \bcl{} are sufficient to avoid loss. In Fig.~\ref{fig:LSurfs}, the surfaces where \epb{} for several \bcl{} magnitudes are shown wherever the splitting is below the hopping threshold $\kappa$. The loss regions are always visible, but they are tuned too far away from the trap center for molecules to access. The striking difference in molecule trap lifetime with and without \bcl{} can be seen in Fig.~\ref{fig:WVplot}a.

\begin{figure}[tb]
\includegraphics[width=\linewidth]{VWFig/vwfig_out.png}%
\caption{
Time traces (a) without bias field (black), with bias field (green dots), and with modulated density (green circles). One body fits (red) give loss rates of $200\text{ s}^{-1}$ without bias field and $2\text{ s}^{-1}$ with full bias field at long times, in agreement with our background gas pressure. At the fixed time $30\text{ ms}$, population is shown as a function of both pin translation and bias field (b), for several values of pin translation, labeled relative to perfect alignment. Fits (red) are calculated by integrating the molecule flux of a thermal ensemble through surfaces where \epb.
\label{fig:WVplot}}
\end{figure}

%allows a full tuning from $200\text{ s}^{-1}$ to complete removal, see 

As a further confirmation of our \cmnt{\epb{}  and $\mu_BB<d_EE$} model of the loss, we translate the magnetic pins along the $\hat{x}$ direction in their mounts to alter the surface where \epb{}, and compare experimental data against expectations (Fig.~\ref{fig:WVplot}b). Qualitatively, this translation serves to disrupt the idealized 2D magnetic quadrupole by adding a small trapping field $\vec{B}\propto B^\prime z\hat{z}$. This means that \bcl{} no longer directly tunes the magnetic field magnitude along the z-axis. Instead, \bcl{} must first overcome the slight trapping field, translating a point of zero field along the $z$-axis and eventually out of the trap. The point of zero magnetic field has $\mu_BB<d_EE$ and lies on the $\phi=\vec{E}\cdot\vec{B}=0$ surface by definition, leading to strong loss unless it is aligned with the trap center, where $E$ is also zero. This means that without \bcl{}, the loss should actually be a local minimum; as $|B_\text{coil}|$ is increased the loss should first worsen and then improve when the zero leaves the trap. This qualitative explanation matches the observed double well structure in population verses \bcl.

Quantitatively, we fit the family of curves (red, Fig.~\ref{fig:WVplot}b) by assuming a time independent thermal population distribution and integrating molecule flux weighted by Landau-Zener probability over the contorted surfaces where \epb{} for each \bcl{} and pin translation. This pure integral computation nonetheless matches the data with only temperature as a free parameter~\footnote{Calculation performed in COMSOL: \href{https://github.com/dreens/spin-flip-integration/}{Source Code}}. \cmnt{The asymmetry of the curves about $B_\text{coil}=0$ comes from a slight shift of the electric quadrupole field caused by an intentional bending of the last pin pair. This bend enhances laser induced fluorescence collection, our detection technique.} The fitted temperature is $170\pm20\text{ mK}$. 

\begin{figure}[tb]
\includegraphics[width=\linewidth]{MWSpec/mwspec_out.png}%
\caption{
Microwave spectroscopy shows that increasing \bcl{} increases population and shifts its distribution to higher energy. Triangles indicate the field at which a molecule can access loss regions for a given \bcl. The dotted lines are eye-guides, but the solid line for highest \bcl{} is a fit to $175\pm15\text{ mK}$
\label{fig:spec}}
\end{figure}

We further confirm our model using microwave spectroscopy (Fig.~\ref{fig:spec}), which detects the molecules experiencing a specific magnetic field by transferring them to an opposite parity state that is dark to our detection laser, as in~\cite{Stuhl2012evap}.\cmnt{, except here we turn off the electric trapping field just before the spectroscopy, which happens before any significant molecular motion. \cmnt{but with a microwave probe directly exciting free space modes of our vacuum chamber\cmnt{instead of a bias tee}.}}  ~Although the magnetic energy is only part of the total energy of molecules in our dual electric and magnetic quadrupole, molecules with a high magnetic energy during our spectroscopic snapshot are more likely to have a high total energy. It is seen that increasing \bcl{} increases population first at low fields and then at higher fields. This is consistent with our calculations showing that loss location moves further from the trap center with increasing \bcl{} (Fig.~\ref{fig:LSurfs}), and thus can be accessed only by higher energy molecules. At largest \bcl{} the distribution fits to $175\pm15\text{ mK}$, agreeing with the pin translation fits (Fig.~\ref{fig:WVplot}b).%Roughly speaking, the average field of $2\text{ kG}$ corresponds to $200\text{ mK}$ for OH. Since this is approximately half the potential energy, we have $U\approx400\text{ mK}$. From the virial theorem for a linear trap, $U = 4.5k_BT$, so we can say the spectrum is consistent with $T\approx 90\text{ mK}$ and consistent with our fitting in fig.~\ref{fig:WVplot}.

%a challenging prospect with our trap deeply integrated in the high voltage decelerator, 

In the case of lowest applied magnetic field in Fig.~\ref{fig:spec}, a negative signal is observed. This indicates a build-up in a negative parity ($|e\rangle$) state. Although the spin-flips we have discussed connect $|f,3/2\rangle$ to $|f,-3/2\rangle$, the latter experiences various avoided crossings due to the large electric confinement fields and remains weakly trapped.  Its field-dressed state character varies, transitioning to $|e,3/2\rangle$ at larger magnetic fields (Fig.~\ref{fig:blocking}d). %nonuniform gradient (repulsive in the center, attractive outside) and a much lower overall trap depth. This secondary state also exhibits spin-flip loss to other lower states, but the enhancement with electric field is related to a quadratic blocking of the Zeeman splitting, and is thus not as dominating as the loss in the primary state due to cubic blocking.

With loss removed, we observe a population trend whose initially fast decay rate decreases over time (Fig.~\ref{fig:WVplot}a, green dots), suggestive of collisions. To test this, we implement a five-fold reduction in initial population and scale the resulting trend by five (green circles). If collisions had contributed, this new trend would show less decay, but we observe no significant change. Our population reduction is achieved early during deceleration, with a microwave field that weakly couples opposite parities, giving a probability for molecules to be lost from the decelerator. This technique should not favor molecules that would end up in one part of the eventually trapped phase-space over another, and so the population distribution should be perturbed only by an overall scaling. Any other perturbation would be unlikely to yield the observed overlap. An alternative hypothesis for the population trend is the existence of chaotic trap orbits with long escape times~\cite{Gonzalez-Ferez2014}.

This absence of collisions differs from our earlier work~\cite{Stuhl2012evap}, which as we have mentioned shows signatures of evaporation for shallow cuts. This could be attributed to a warmer initial temperature and correspondingly lower density, reduced molecule number, or differences in trap geometry and loading. \cmnt{It is also possible that the spin-flip loss is playing a role even for the light evaporation cuts, although it is unclear how this could masquerade as a normalized low field density enhancement.} Moving forward, we aim to continue with the spin-flip loss removed but increase the density so as to reattain the collisional regime, by means of several improvements~\cite{Even2015,Segev2017}.

%~\footnote{We use microwaves to couple electrically weak and strong field seekers during the first half of deceleration. Spatial inhomogeneities on the order of the cloud size are unlikely given the $15\text{ cm}$ microwave wavelength, and any that exist are remixed during the remaining deceleration.} The trend remains similar with this density-modulation, suggesting that single-particle physics is chiefly responsible. This contrast with our evaporation work~\cite{Stuhl2012evap} could be attributable to our warmer initial temperature, decreased molecule number, or other changes in geometry or trap-loading. We hope soon to implement several density increasing measures and return to the collisional regime. The slowly varying but single-body decay rate could be related to the existence of high energy chaotic orbits with long escape times, as seen in other exotically shaped trapping potentials~\cite{Gonzalez-Ferez2014}.

% Next: edit conclusion.
Molecule enhanced spin-flip loss arises in mixed electric and magnetic fields due to a competition between field quantization axes where \epb{} and $\mu_BB<d_EE$. We conclusively demonstrate this effect and overcome it using our dual magnetic and electric quadrupole trap. Our explanation of the effect provides detailed predictions of how its location and magnitude ought to scale with bias field and trap alignment, which we experimentally verify. Our results correct existing predictions about molecular spin-flips in mixed fields and pave the way toward further improvements in molecule trapping and cooling.

We acknowledge the Gordon and Betty Moore Foundation, the ARO-MURI, JILA PFC, and NIST for their financial support. T.L. acknowledges support from the Alexander von Humboldt Foundation through a Feodor Lynen Fellowship. We thank J.L. Bohn and S.Y.T. van de Meerakker for helpful discussions. We thank Goulven Qu\'em\'ener for his continued involvement in this research.

%includes uncited bib entries
%\nocite{*}
\bibliographystyle{apsrev4-1_no_Arxiv}
\bibliography{MolecularSpinFlipLoss}% Produces the bibliography via BibTeX.

\end{document}
%
% ****** End of file MolecularMajoranaLoss.tex ******
